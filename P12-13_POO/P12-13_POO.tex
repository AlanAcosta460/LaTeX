\documentclass[11pt, twocolumn]{article}

\usepackage[spanish]{babel}
\usepackage[none]{hyphenat}
\usepackage[left=1.2cm, right=1.2cm, top = 2cm, bottom=2.5cm]{geometry}
% \usepackage{setspace}
\usepackage{parskip}
\usepackage[export]{adjustbox}
\usepackage{enumitem}
\usepackage{listings}
\usepackage[dvipsnames]{xcolor}
\usepackage{fancyhdr}
\usepackage{graphicx}
\usepackage{caption}
% \usepackage{subcaption}
% \usepackage{wrapfig}
% \usepackage{multirow, makecell}
% \usepackage{float}
% \usepackage{amsmath} 
% \usepackage{amsfonts}
\usepackage[hidelinks]{hyperref}
\usepackage{csquotes}

\newcommand{\linejump}{\hfill \break}
\renewcommand{\thefootnote}{\fnsymbol{footnote}}
% \newcommand{\unit}[1]{\ensuremath{\, \mathrm{#1}}}

\definecolor{dkgreen}{rgb}{0,0.6,0}
\definecolor{gray}{rgb}{0.5,0.5,0.5}
\definecolor{mauve}{rgb}{0.58,0,0.82}
\lstset{
  language=Java,
  aboveskip=3mm,
  belowskip=3mm,
  showstringspaces=false,
  columns=flexible,
  basicstyle={\tiny\ttfamily},
  numbers=none,
  numberstyle=\tiny\color{gray},
  keywordstyle=\color{blue},
  commentstyle=\color{dkgreen},
  stringstyle=\color{mauve},
  breaklines=true,
  breakatwhitespace=true,
  tabsize=2
}

\sloppy
\setlength{\parindent}{0cm}
\setlength{\columnsep}{0.5cm}
\decimalpoint
\graphicspath{{img/}}

\hypersetup{colorlinks=true, urlcolor=blue, citecolor=blue}
\urlstyle{same}

\pagestyle{fancyplain}
\fancyhf{}
\fancyhead[L]{\scriptsize 
  Universidad Nacional Autónoma de México \\
  Laboratorio de Programación Orientada a Objetos \\
  M.C. Leonardo Ledesma Dominguez
}
\fancyhead[R]{\thepage}

\begin{document}
  \twocolumn[
    \centering
    Acosta Porcayo Alan Omar, Gutiérrez Grimaldo Alejandro, Medina Villa Samuel

    \linejump

    \textbf{\LARGE{Práctica 12. Hilos}} \\
    
    \linejump
  ]
      
  \footnotetext{
    \scriptsize 
    Acosta Porcayo Alan Omar Ing. en Computación 320206102 \\
    Gutiérrez Grimaldo Alejandro Ing. en Computación 320282098 \\
    Medina Villa Samuel Ing. en Computación 320249538
  }
        
  \fancyfoot{}

  \section*{Resumen}
  

  \section*{Introducción}
  

  \section*{Objetivos}
  \begin{itemize}
    \item 
  \end{itemize}

  \section*{Metodología}
  \textit{\textbf{Hilo.java}}
  \begin{lstlisting}
public class Hilo extends Thread{
	public Hilo(String nombre){
		super(nombre);
	}
	public void run(){
		for(int i= 0; i<5; i++){
			System.out.println("iteracion: " + (i+1) + "de" + getName());
		}
		System.out.println("Termina el " + getName());

	}

	public static void main(String[] args) {
		new Hilo ("Primer hilo").start();
		new Hilo ("Segundo hilo").start();
		System.out.println("Termina el hilo principal");
		
	}
}
  \end{lstlisting}

  \textit{\textbf{Cuenta.java}}
  \begin{lstlisting}
class Cuenta extends Thread{
	private static long saldo = 0;
	public Cuenta(String nombre){
		super(nombre);

	}

	public void run(){
		if (getName().equals("Deposito 1") || getName().equals("Deposito 2")){
			this.depositarDinero(100);
		}else{
			this.extraerDinero(50);
		}
	}

	public synchronized void extraerDinero(int cantidad){
		try{

			if(saldo <= 0){
				System.out.println(getName() + "espera a que depositen lana");
				sleep(5000);

			}

		}catch(InterruptedException e){
			System.out.println(e);

		}

		saldo-= cantidad;
		System.out.println(getName() + "extrajo 50, el saldo es: " + saldo);
		notifyAll();

	}

	public synchronized void depositarDinero(int cantidad){
		saldo += cantidad;
		System.out.println("Se deposito dinero (100), el saldo es: " + saldo );
		notifyAll();

	}
	public static void main(String[] args) {
		new Cuenta("Acceso A").start();
		new Cuenta("Acceso B").start();
		new Cuenta("Deposito 1 ").start();
		new Cuenta("Deposito 2 ").start();
		System.out.println("termina el main");
	}
}
  \end{lstlisting}

  \textit{\textbf{Grupo.java}}
  \begin{lstlisting}
public class Grupo extends Thread{
	public Grupo(ThreadGroup G, String n){
		super(G,n);

	}
	public void run(){
		for (int i=0;i<10 ;i++ ) {
			System.out.println(getName());
			
		}
	}

	public static void listarHilos(ThreadGroup grupoActual){
		int numhilos;
		Thread[]listaHilos;

		numhilos = grupoActual.activeCount();
		listaHilos = new Thread[numhilos];
		System.out.println("Numero de hilos activos en el grupo " + numhilos);
		for (int i=0; i<numhilos ;i++ ) {
			System.out.println("Hilo Activo: " + (i+1) + "=" + listaHilos[i].getName());
			
		}
	
	}

	public static void main(String[] args) {

		ThreadGroup grupoH = new ThreadGroup("Grupo de Hilos prioridad normal");
		Thread hilo1 = new Grupo(grupoH, "Hilo 1 con prioridad normal");
		Thread hilo2 = new Grupo(grupoH, "Hilo 2 con prioridad normal");
		Thread hilo3 = new Grupo(grupoH, "Hilo 3 con prioridad normal");
		Thread hilo4 = new Grupo(grupoH, "Hilo 4 con prioridad normal");
		Thread hilo5 = new Grupo(grupoH, "Hilo 5 con prioridad normal");
		Thread hilo6 = new Grupo(grupoH, "Hilo 6 con prioridad normal");
		Thread hilo7 = new Grupo(grupoH, "Hilo 7 con prioridad normal");
		Thread hilo8 = new Grupo(grupoH, "Hilo 8 con prioridad normal");
		Thread hilo9 = new Grupo(grupoH, "Hilo 9 con prioridad normal");
		Thread hilo10 = new Grupo(grupoH, "Hilo 10 con prioridad normal");

		hilo7.setPriority(Thread.MAX_PRIORITY);
		hilo10.setPriority(Thread.MIN_PRIORITY);
		grupoH.setMaxPriority(Thread.NORM_PRIORITY);

		System.out.println("La prioridad del grupo es de : " + grupoH.getMaxPriority());

		System.out.println("La prioridad del Thread es de : " + hilo1.getPriority());
		System.out.println("La prioridad del Thread es de : " + hilo2.getPriority());
		System.out.println("La prioridad del Thread es de : " + hilo3.getPriority());
		System.out.println("La prioridad del Thread es de : " + hilo4.getPriority());
		System.out.println("La prioridad del Thread es de : " + hilo5.getPriority());
		System.out.println("La prioridad del Thread es de : " + hilo6.getPriority());
		System.out.println("La prioridad del Thread es de : " + hilo7.getPriority());
		System.out.println("La prioridad del Thread es de : " + hilo8.getPriority());
		System.out.println("La prioridad del Thread es de : " + hilo9.getPriority());
		System.out.println("La prioridad del Thread es de : " + hilo10.getPriority());

		hilo1.start();
		hilo2.start();
		hilo3.start();
		hilo4.start();
		hilo5.start();
		hilo6.start();
		hilo7.start();
		hilo8.start();
		hilo9.start();
		hilo10.start();

		listarHilos(grupoH);
			
	}
	
}
  \end{lstlisting}

  \section*{Resultados}
  \subsection*{Problema 1}
  

  \section*{Conclusiones}
  

  \twocolumn[
    \centering
    Acosta Porcayo Alan Omar, Gutiérrez Grimaldo Alejandro, Medina Villa Samuel

    \linejump
    
    \textbf{\LARGE{Práctica 13. Patrones de diseño}} \\

    \linejump
  ]

  \section*{Resumen}
  

  \section*{Introducción}
  

  \section*{Objetivos}
  \begin{itemize}
    \item   
  \end{itemize}

  \section*{Resultados}
  \subsection*{Problema 1}
  

  \section*{Conclusiones}
  

  \section*{Referencias}
  \begin{small}
    Solano, J. (2017, 20 enero). \textit{Manual de prácticas de Programación Orientada a Objetos}. Laboratorio de Computación Salas A y B. \url{http://lcp02.fi-b.unam.mx/} \\
  \end{small}
\end{document}