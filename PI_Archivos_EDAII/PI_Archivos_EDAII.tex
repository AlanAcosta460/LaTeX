\documentclass[11pt, twocolumn]{article}

\usepackage[spanish]{babel}
\usepackage[none]{hyphenat}
\usepackage[
  left=2.5cm,
  right=2.5cm,
  top=2.5cm,
  bottom=2.5cm,
  paperwidth=21cm,
  paperheight=29.7cm, 
]{geometry}
\usepackage{setspace}
\usepackage{parskip}
\usepackage[export]{adjustbox}
\usepackage{enumitem}
\usepackage{listingsutf8}
% \usepackage[dvipsnames]{xcolor}
\usepackage{fancyhdr}
\usepackage{graphicx}
\usepackage{caption}
% \usepackage{subcaption}
% \usepackage{wrapfig}
% \usepackage{multirow, makecell}
% \usepackage{float}
% \usepackage{amsmath} 
% \usepackage{amsfonts}
\usepackage[hidelinks]{hyperref}
\usepackage{csquotes}
\usepackage{lipsum}
\usepackage{fontspec}

\newcommand{\linejump}{\hfill \break}
% \newcommand{\unit}[1]{\ensuremath{\, \mathrm{#1}}}
\renewcommand{\thefootnote}{\fnsymbol{footnote}}
\renewcommand\thesubsection{\Roman{subsection}}

\sloppy
\onehalfspacing
\setmainfont{Georgia}
\setlength{\parindent}{0.5cm}
\setlength{\columnsep}{0.5cm}
\decimalpoint
\graphicspath{{img/}}
\hypersetup{colorlinks=true, urlcolor=blue, citecolor=blue}
\urlstyle{same}

\begin{document}
  \twocolumn[
    \centering
    \textbf{\huge{Archivos}} \\

    \linejump

    A. O. Acosta Porcayo, A. Gutiérrez Grimaldo, S. Medina Villa, A. A. Uribe Urieta

    \linejump
  ]
  
  \textbf{Resumen.} \lipsum[1]

  \textbf{Palabras Clave.} \textit{Archivos, Organización de archivos, Acceso a archivos, Sistema de archivos.}

  \section{INTRODUCCIÓN}
  Los archivos son una parte esencial de nuestra vida cotidiana en la era en la que nos situamos y tienen una importancia fundamental por diversas razones, pues son la forma más común de almacenar información de manera digital. En este artículo, exploraremos a detalle qué son los archivos y cómo funcionan, cómo es su organización y su acceso, así como los sistemas de archivos que los organizan y gestionan.

  \section{CONTENIDO}
  \subsection{Definición y operaciones}
  Un archivo es una unidad de almacenamiento digital que contiene datos, información o programas. Puede contener texto, imágenes, videos, música o cualquier tipo de información digital. Los archivos se utilizan para almacenar, organizar y acceder a datos de manera eficiente. Cada archivo tiene un nombre único que lo identifica y una extensión que indica su tipo. Por ejemplo, un documento de texto puede tener el nombre ``mi$\_$documento'' y la extensión del mismo ``.txt''.

  Las operaciones básicas de los archivos son las siguientes.

  \textbf{Creación:} Para crear un archivo, en primer lugar, se asigna un nombre y una ubicación de ruta en el sistema de archivos. Después, se pueden escribir o copiar datos en el archivo.

  \textbf{Apertura:} La apertura de un archivo es importante y se emplea ya sea para consultar su contenido o para actualizarlo, es imprescindible llevar a cabo la acción de abrirlo. Esta acción debe realizarse de manera previa a cualquier operación de lectura o escritura.

  \textbf{Lectura (consulta):} La operación de lectura implica acceder a la información contenida de un archivo y recuperar el contenido almacenado en él. Los archivos se pueden leer para mostrar su contenido en pantalla o procesarlos de alguna manera.

  \textbf{Escritura (modificación):} Es la operación de agregar, modificar o actualizar datos en un archivo. La escritura permite editar o ampliar la información contenida en un archivo.

  \textbf{Eliminación:} La eliminación de un archivo implica su borrado permanente del sistema de archivos. Esto se hace para liberar espacio en el almacenamiento y eliminar datos innecesarios.

  \textbf{Copia y Movimiento:} Los archivos se pueden copiar o mover a otras ubicaciones dentro del sistema de archivos. Esto es útil para respaldar datos o reorganizar la información.

  \textbf{Renombrado:} Cambiar el nombre de un archivo es una operación común para mejorar la organización de los archivos.

  \subsection{Organización de archivos (Físico y lógico)}
  La configuración de un archivo establece la manera en que los datos se distribuyen en el medio de almacenamiento, o se puede describir la organización como la manera en que los datos se organizan en un archivo. Existen dos aspectos principales de la organización de archivos: organización física y la organización lógica.

  \subsubsection*{Organización física}
  La disposición física de archivos hace referencia a la manera en que los datos se guardan en medios de almacenamiento tangibles.

  \textbf{Secuencial:} Un archivo organizado de forma secuencial es una secuencia de registros almacenados uno tras otro en un medio de almacenamiento externo, de manera que para acceder a un registro específico n, es necesario recorrer todos los n-1 registros que lo anteceden.

  \textbf{Aleatoria o directa:} Un archivo con organización directa se caracteriza por la falta de correspondencia entre el orden físico y el orden lógico de los datos. Los datos se ubican en el archivo y se accede a ellos de manera aleatoria a través de su posición. Esta forma de organización ofrece la ventaja de permitir la lectura y escritura de registros en cualquier orden y posición. Sin embargo, presenta el inconveniente de requerir una programación que relacione el contenido de un registro con su ubicación, lo que implica la posibilidad de que haya espacios vacíos en el soporte de almacenamiento.

  \textbf{Indexado:} Un archivo secuencial indexado ofrece una combinación de opciones de acceso que combina las características de un archivo secuencial con las de un archivo relativo o de acceso directo. Se utiliza una tabla que enumera de manera secuencial los valores de clave del archivo y, para cada uno de ellos, proporciona la dirección del registro correspondiente.

  \subsubsection*{Organización lógica}
  La organización lógica de archivos hace referencia a la manera en que los datos son organizados y guardados en un archivo, considerando la relación y disposición de los registros o datos en función de su contenido y finalidad. Esta organización se centra en la forma en que los datos son estructurados y almacenados para simplificar su posterior acceso y manipulación. Algunos métodos son:

  \textbf{Registros:} La organización de datos en registros es una práctica común. Estos registros son estructuras de datos que albergan campos de información relacionados. Es importante destacar que los registros pueden tener una estructura que varía, lo que significa que algunos podrían tener un formato fijo, mientras que otros pueden contar con un formato variable, adaptándose a las necesidades específicas del sistema.

  \textbf{Campos:} Dentro de cada registro, se encuentran los campos, que son las unidades de datos individuales. Cada campo tiene un tipo de dato específico, como números enteros, cadenas de texto, fechas, entre otros. Esto permite la clasificación y organización precisa de la información, facilitando su manipulación y recuperación.

  \textbf{Clave de Búsqueda:} La clave de búsqueda es un elemento crítico en la organización lógica de archivos. Se trata de un campo único o un conjunto de campos que se utilizan como punto de acceso para recuperar registros específicos en el archivo. La elección de la clave de búsqueda adecuada es esencial para agilizar la búsqueda y recuperación de datos.

  \textbf{Índices:} Los índices desempeñan un papel importante en la optimización de la búsqueda de registros en un archivo. Pueden ser de dos tipos principales: índices secundarios y primarios. Los índices secundarios se basan en valores de campo que no son la clave de búsqueda, y los índices primarios se basan en la clave de búsqueda misma. Estos índices permiten acelerar el acceso a la información al proporcionar un mapa de ubicación eficiente de los registros en el archivo.

  \subsection{Acceso a archivos (Físico y lógico)}
  \lipsum[8-9]

  \subsection{Sistema de archivos}
  Los archivos no existen en un vacío; están organizados y gestionados por sistemas de archivos. Por lo tanto, un sistema de archivos es un conjunto de estructuras y de reglas que permiten el almacenamiento, la recuperación y la gestión de archivos en un dispositivo de almacenamiento, como los pueden ser un disco duro, una memoria USB o una tarjeta de memoria, etc. Por lo tanto, corresponde a un sistema de almacenamiento de un dispositivo de memoria, que estructura y organiza la escritura, búsqueda, lectura, almacenamiento, edición y eliminación de archivos de una manera concreta.

  Cada sistema de archivos cuenta con sus propias características y algunas limitaciones. Entre algunos de los sistemas de archivos más comunes están:

  \begin{itemize}
    \item \textbf{FAT (File Allocation Table):} Utilizado principalmente en dispositivos de almacenamiento USB y tarjetas de memoria. Es un sistema de archivos simple pero ampliamente compatible.
    \item \textbf{NTFS (New Technology File System):} Utilizado en sistemas Windows, ofrece características avanzadas como permisos de acceso y compresión de archivos.
    \item \textbf{ext4 (cuarta versión extendida):} Común en sistemas Linux, ext4 es eficiente y confiable, con soporte para archivos grandes y sistemas de archivos de gran capacidad.
    \item	\textbf{HFS+ (Hierarchical File System Plus):} Utilizado en sistemas macOS, es conocido por su manejo de archivos de gran tamaño y metadatos avanzados.
    \item \textbf{APFS (Apple File System):} El sistema de archivos más reciente de Apple, diseñado para mejorar la eficiencia y la seguridad en dispositivos macOS y iOS.
    \item \textbf{NTFS (Network File System):} Un sistema de archivos de red utilizado en sistemas Unix y Linux, que permite compartir archivos y recursos en redes.
  \end{itemize}
  
  Los sistemas de archivos son esenciales para garantizar que los archivos se almacenan de manera eficiente, se puedan acceder de manera rápida y se mantengan seguros. La elección del sistema de archivos adecuado depende de las necesidades específicas del usuario y del entorno en el que se utiliza. Comprender cómo funcionan los archivos y cómo se organizan en sistemas de archivos es esencial para maximizar la eficiencia y la seguridad en la gestión de datos en cualquier dispositivo de almacenamiento.


  \section{CONCLUSIONES}
  \lipsum[12-13]

  \section{REFERENCIAS}
  \setlength{\parindent}{0.0cm}
  \small
  \textit{Archivo Secuencial-Indexado.} (s. f.). Administracion de Archivos. \url{https://nlaredo.tecnm.mx/takeyas/Apuntes/Administracion_Archivos/Apuntes/Archivos_secuencial-indexados.PDF} \\

  Gálvez, S. Mora, M. (2005). \textit{Java a Tope: Traductores Y Compiladores Con Lex/yacc, Jflex/cup Y Javacc.} Universidad de Málaga. \\

  \textit{IBM documentation.} (s. f.). \url{https://www.ibm.com/docs/es/cobol-linux-x86/1.2?topic=clause-file-organization} \\

  López, J. (2011). \textit{Administración de sistemas operativos: Un enfoque práctico.} Editorial Ra-Ma. \\

  Silberschatz, A. Galvin, P. Gagne, G. (2008). \textit{Operating System Concepts.} John Wiley $\&$ Sons Inc. \\

  Tanenbaum, A. Bos, H. (2017). \textit{Modern Operating Systems.} Pearson. \\

  Universidad Autonoma del Estado de Mexico. (s. f.). \textit{Organizacion de Archivos} [Diapositivas]. uamex. \url{http://ri.uaemex.mx/oca/view/20.500.11799/34751/1/secme-19554.pdf} \\

  Universidad Nacional Autonoma de Mexico. (s. f.). \textit{Tipos de Archivos} [Diapositivas]. \url{http://fcasua.contad.unam.mx/apuntes/interiores/docs/98/3/informatica3.pdf} \\
\end{document}