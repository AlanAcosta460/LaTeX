\documentclass[12pt]{article}

\usepackage[spanish]{babel}
\usepackage[none]{hyphenat}
\usepackage{helvet}
\usepackage{setspace}
\usepackage[margin=3cm]{geometry}
\usepackage{fancyhdr}

\renewcommand{\familydefault}{\sfdefault}

\sloppy
\onehalfspacing
\pagestyle{fancy}
\fancyhead[R]{Alan Omar Acosta Porcayo\\RETI 18}
\setlength{\headheight}{27.69403pt}
\decimalpoint

\begin{document}
    \begin{center}
        \textbf{Apoyo para la exposición}
    \end{center}
    \begin{itemize}
        \item La Ciudad de México tiene 9 millones de habitantes, según la INEGI. 
        \item La movilidad es fundamental para el crecimiento de la economía. Esta necesidad genera un problema de movilidad que ha aumentado con los años.
        \item A pesar de que esta situación se ha convertido en la rutina, es importante estudiar el fenómeno para lograr solucionarlo.
        \item Crecimiento del parque vehicular de la Zona Metropolitana de la Ciudad de México (ZMCM), de 3.5 millones en el año 2000 a 9.5 millones en el 2015. Esto es provocado por la necesidad de movilidad de la poblacional y la distribución los productos.
        \item Un factor que si puede ser solucionado es la cantidad de información de los usuarios. Los usuarios deciden cómo y cuando recorrer la que consideren la mejor ruta. Su decisión es influenciada por criterios como el costo, tiempo seguridad y comodidad. 
        \item Las malas decisiones del usuario afectan a otros como consecuencia.
        \item Un mal diseño de la infraestructura y el uso de controladores de tránsito obsoletos son causas principales del problema de transporte. La ingeniera del tránsito se enfoca en la optimización del flujo vehicular.
        \item Hay vialidades que presentan un peligro potencial en la circulación.
        \item La intersección a nivel es una área que se comparte por dos o más caminos, es fácil de planear y construir pero sino tiene la capacidad de servicio suficiente, congestiona el flujo vehicular en las horas pico.
        \item Otro tipo de intersección es la glorieta, caracterizada por una circulación circular pero que debido a la indecisión y a la incomprensión de las reglas de conducción, causando conflictos y accidentes de baja gravedad, pero que obstruyen el flujo normal del tráfico.
        \item Para minimizar la cantidad de accidentes automovilísticos se utiliza un sistema de semáforos que controlan el tránsito y la seguridad.
        \item Sin embargo, es posible que las fallas en los semáforos provoquen confusión y atascamiento en zonas muy transitadas.
        \item Por otro lado, el análisis de las causas de los accidente vehiculares es realizado por una rama de la psicología llama psicología del tránsito.
        \item Según esta área de estudio, algunos factores que afectan el desempeño de los conductores son el sueño, la fatiga y el consumo de sustancias como el alcohol. De tal manera que generan comportamientos de riesgo como la agresión y la hostilidad hacia otros conductores y peatones.
        \item Una alternativa que viene a la mente para solucionar el problema del tráfico vehicular es la de expandir o agregar las vías de tránsito, pero esto solo puede llegar a empeorar la situación.
        \item Según la paradoja de Brases, agregar una nueva vialidad a la red de trasportarse no necesariamente conlleva una mejora en el sentido de reducción del tiempo total de viaje.
        \item También nombrada como el tráfico inducido, esta situación tan poco intuitiva se explica bajo el modelo económico de la oferta y demanda. 
        \item Según un estudio realizado con datos de 1950 a 2001, la existencia del tráfico inducido en México se comprueba debido a que la reducción del tiempo de transporte se traduce en un aumento de las distancias recorridas. 
        \item Además del estudio del tráfico sistemas viales complejos como los de las ciudades, las causas de este problema también se han identificado en sistemas a pequeña escala.
        \item Los resultados de un estudio realizado en Ciudad Universitaria en 2018, centrados principalmente el sistema de transporte PumaBus, muestran las causas más relevantes de las demoras.
        \item El motivo más relevante es el tiempo de ascenso y descenso de pasajeros, representando entre un 66\% y 74\% del total de demoras. 
        \item Por otro lado, la velocidad promedio de los autobuses es de 25.3 km/h, que es mucho menor a la permitida para la libre circulación que es de 40 km/h, significando una reducción de un 23\% en la eficiencia.
        \item Las detenciones por semáforos y la congestión representan el segundo y tercer motivo de demora respectivamente. El horario que especialmente presento problemas de congestión fue el de la tarde y valle-tarde, debido a que los usuarios tienen a trasportarse en sus vehículos a lugares de almuerzo. 
        \item Otros motivos con menor repercusión fueron la congestión por los estacionamientos, por intersección o por imprudencia del peatón. Los dos primeros solo ocurriendo en horas pico.
        \item Algunas propuestas realistas requieren de una mejora en los dispositivos de control, como la instalación de circuitos cerrados de televisión, que permitan reducir los riesgos que se enfrentan en entornos sociales como el de tráfico vehicular. 
        \item Asimismo, la implementación de las tecnologías basadas en la inteligencia artificial que provean información y optimicen el flujo dependiendo de la densidad vehicular en tiempo real.
        \item Es posible que con el desarrollo de nueva políticas gubernamentales y con una investigación más extensa de las causas se logren propuestas y acciones que solucionen o, en menor medida, paren el crecimiento paulatino del tráfico vehicular.
    \end{itemize}
\end{document}