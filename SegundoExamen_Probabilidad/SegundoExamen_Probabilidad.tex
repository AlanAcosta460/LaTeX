\documentclass[12pt]{report}

\usepackage[spanish]{babel}
\usepackage[none]{hyphenat}
\usepackage[left=1.5cm, right=1.5cm, top = 2cm, bottom=2.5cm]{geometry}
\usepackage{parskip}
\usepackage[export]{adjustbox}
\usepackage{enumitem}[shortlabels]
\usepackage{graphicx}
\usepackage{caption} 
\usepackage{wrapfig}
\usepackage{xcolor}
\usepackage{longtable}
\usepackage{multirow, makecell}
\usepackage{amsmath} 
\usepackage{amsfonts}
\usepackage[hidelinks]{hyperref}
\usepackage{csquotes}
\usepackage{helvet}
\usepackage{pgfplots}

\newcommand{\linejump}{\hfill \break}
\renewcommand{\familydefault}{\sfdefault}

\definecolor{azuloscuro}{RGB}{0,0,128}

\sloppy
\decimalpoint
\graphicspath{{img/}}

\begin{document}
  \noindent
  \begin{center}
    \textbf{UNIVERSIDAD NACIONAL AUTÓNOMA DE MÉXICO \\
    FACULTAD DE INGENIERÍA \\
    DIVISIÓN DE CIENCIAS BÁSICAS \\
    COORDINACIÓN DE CIENCIAS APLICADAS \\
    SEGUNDO EXAMEN PARCIAL DE PROBABILIDAD}
  \end{center}

  \linejump

  \textbf{Profesora} Act. Nora Patricia Rocha Miller \hfill \textbf{15 de noviembre de 2023} \\
  \textbf{Semestre 2024-1} \hfill \textbf{FECHA DE ENTREGA: 16/11/23, 23:00 horas}

  \linejump

  \textbf{Nombre:} \underline{\hspace*{2.8cm} Acosta \hspace*{2.8cm} Porcayo \hspace*{2.8cm} Alan Omar \hspace*{2.8cm}}  \\
  \hspace*{3.7cm} \textbf{Apellido paterno} \hspace*{0.75cm} \textbf{Apellido materno} \hspace*{1.8cm} \textbf{Nombre(s)}

  \linejump

  {\color{azuloscuro} Instrucciones: Lea cuidadosamente el examen antes de resolverlo. Identifique el experimento aleatorio, la variable aleatoria muestre el desarrollo completo de los que se pide utilizando la notación matemática correspondiente.}

  \linejump
  \linejump

  \begin{enumerate}
    \item Muchos productos se fabrican en masa en líneas de ensamble automatizadas. La distribución de probabilidad del tiempo entre dos llegadas sucesivas de componentes fabricadas a la salida de la línea de ensamble tiene la siguiente función de densidad

    \[
      f(x) = \left\{
      \begin{array}{ll}
          \dfrac{e^{-.05x}}{20}, & \hspace*{1cm} x > 0 \\
          0 & \hspace*{1cm} \text{en otro caso}
      \end{array}
      \right.
    \]

  \linejump
  \begin{enumerate}[label=\alph*.]
    \item ¿Qué mide la variable aleatoria $X$?

    $X$ mide el tiempo entre la salida sucesiva de dos componentes.

    \newpage
    \item Grafique la función de densidad
    
    \begin{center}
      \begin{tikzpicture}
          \begin{axis}[
            axis lines=middle,
            xlabel=$x$,
            ylabel=$f(x)$,
            xmin=0,
            ymin=0,
            grid,
          ]
          
          \addplot[
            domain=0:10,
            samples=100,
            color=blue,
            thick,
          ]
          {exp(-0.05*x)/20};
        \end{axis}
      \end{tikzpicture}
    \end{center}
    
    \item ¿Qué probabilidad hay de que un tiempo entre llegadas (el tiempo entre llegadas de dos tubos de magnetrón) sea menor de 10 segundos?
    
    \begin{align*}
      P(X < 10) &= \int_0^{10} f(x)dx = \int_0^{10} \frac{e^{-0.05x}}{20} dx = \frac{1}{20} \int_0^{10} e^{-0.05x} dx = \frac{1}{20} \left[ \frac{e^{-0.05x}}{-0.05} \right]_0^{10} \\
      &= \left[ -e^{-0.05x} \right]_0^{10} = -e^{-0.05(10)} - (-e^{-0.05(0)}) = -e^{-0.5} - (-e^0) \\
      &= -e^{-0.5} + 1 = 1 - 0.6065 = {\color{blue}\boxed{0.3935}}
    \end{align*}

    \item ¿Qué probabilidad hay de que los siguientes cuatro tiempos entre llegadas sean todos menores de 10 segundos?
    
    \begin{align*}
      P(X_1 < 10 \cap X_2 < 10 \cap X_3 < 10 \cap X_4 < 10) &= P(X_1 < 10) \cdot P(X_2 < 10) \cdot P(X_3 < 10) \\
      & \cdot P(X_4 < 10) = P(X < 10)^4 \\ 
      &= 0.3935^4 \approx {\color{blue}\boxed{0.0240}}
    \end{align*}
    
    \item Encuentre la Función de Distribución Acumulada y grafíquela
    
    \begin{align*}
      F(x) &= \int_{-\infty}^x f(t)dt = \int_{-\infty}^0 0dt + \int_0^x \frac{e^{-0.05t}}{20} dt = \frac{1}{20} \int_0^x e^{-0.05t} dt = \frac{1}{20} \left[ \frac{e^{-0.05t}}{-0.05} \right]_0^x \\ 
      &= \left[ -e^{-0.5t} \right]_0^x = -e^{-0.05x} - (-e^0) = -e^{-0.05x} + 1 = {\color{blue}\boxed{1 - e^{-0.05x}}}
    \end{align*}

    \[
      F(x) = \left\{
      \begin{array}{ll}
          1 - e^{-0.05x}, & \hspace*{1cm} x > 0 \\
          0 & \hspace*{1cm} \text{en otro caso}
      \end{array}
      \right.
    \]

    \begin{center}
      \begin{tikzpicture}
          \begin{axis}[
            axis lines=middle,
            xlabel=$x$,
            ylabel=$f(x)$,
            xmin=0,
            ymin=0,
            grid,
          ]
          
          \addplot[
            domain=0:200,
            samples=100,
            color=blue,
            thick,
          ]
          {1- exp(-0.05*x)};
        \end{axis}
      \end{tikzpicture}
    \end{center}

    \item Utilice la FDA del inciso anterior y calcule las siguientes probabilidades
    
    \begin{enumerate}[label=\alph*)]
      \item Que el tiempo entre llegadas exceda el minuto
      
      \begin{align*}
        P(X > 60) &= P(60 < X < \infty) = F(\infty) - F(60) = 1 - F(60) \\
        &= 1 - (1 - e^{-0.05(60)}) = e^{-3} \approx {\color{blue}\boxed{0.0498}}
      \end{align*}

      \item Si ya excedió 10 segundos, exceda 20 segundos
      
      \begin{align*}
        P(X > 20 \mid X > 10) &= P(20 < X < \infty \mid 10 < X < \infty) \\ 
        &= \frac{P(20 < X < \infty \cap 10 < X < \infty)}{P(10 < X < \infty)} \\
        &= \frac{P(20 < X < \infty)}{P(10 < X < \infty)} \\
        &= \frac{F(\infty) - F(20)}{F(\infty) - F(10)} = \frac{1 - F(20)}{1 - F(10)} \\
        &= \frac{1 - (1 - e^{-0.05(20)})}{1 - (1 - e^{-0.05(10)})} = \frac{e^{-1}}{e^{-0.5}} = \frac{0.3679}{0.6065} \approx {\color{blue}\boxed{0.6066}}
      \end{align*}
    \end{enumerate}

    \item Calcule el tiempo medio entre dos llegadas sucesivas

    \begin{align*}
      E[X] &= \int_{-\infty}^{\infty} xf(x)dx = \int_{-\infty}^0 0dx + \int_0^{\infty} \frac{xe^{-0.05x}}{20} dx = \frac{1}{20} \int_0^{\infty} xe^{-0.05x} dx 
    \end{align*}

    \begin{align*}
      u = x \hspace*{3cm} & dv = e^{-0.05x} dx \\
      du = dx \hspace*{3cm} & v = \frac{e^{-0.05x}}{-0.05} 
    \end{align*}

    \begin{align*}
      E[X] &= \frac{1}{20} \left[ \frac{xe^{-0.05x}}{-0.05} \right]_0^{\infty} - \frac{1}{20} \int_0^{\infty} \frac{e^{-0.05x}}{-0.05} dx = \left[ -xe^{-0.05x} \right]_0^\infty + \int_0^\infty e^{-0.05x} dx \\ 
      &= \left[ -xe^{-0.05x} \right]_0^\infty + \left[ \frac{e^{-0.05x}}{-0.05} \right]_0^\infty = \left[ -\infty e^{-\infty} - (-0e^0) \right] + \left[ \frac{e^{-\infty}}{-0.05} - \frac{e^0}{-0.05} \right] \\
      &= \left[ 0 + 0 \right] + \left[ 0 - \frac{1}{-0.05} \right] = \frac{1}{0.05} = {\color{blue}\boxed{20}}
    \end{align*}

    \item Calcule la varianza de $X$
    
    \begin{align*}
      \sigma^2 &= E\left[ \left( X - E[X] \right)^2 \right] = E[X^2] - E[X]^2 = \int_{-\infty}^{\infty} x^2 f(x)dx - E[X]^2 \\
      &= \int_{-\infty}^0 0dx + \int_0^{\infty} \frac{x^2e^{-0.05x}}{20} dx - 20^2 = \frac{1}{20} \int_0^{\infty} x^2e^{-0.05x} dx - 400
    \end{align*}

    \begin{align*}
      u = x^2 \hspace*{3cm} & dv = e^{-0.05x} dx \\
      du = 2xdx \hspace*{3cm} & v = \frac{e^{-0.05x}}{-0.05}
    \end{align*}

    \begin{align*}
      \sigma^2 &= \frac{1}{20} \left[ \frac{x^2e^{-0.05x}}{-0.05} \right]_0^{\infty} - \frac{1}{20} \int_0^{\infty} \frac{2xe^{-0.05x}}{-0.05} dx - 400 \\
      &= \left[ -x^2e^{-0.05x} \right]_0^\infty + 2\int_0^\infty xe^{-0.05x} dx - 400 \\
      &= \left[ -\infty e^{-\infty} - (-0e^0) \right] + 2\left[ 20(E[X]) \right] - 400 \\
      &= \left[ 0 + 0 \right] + 2\left[ 20(20) \right] - 400 = 800 - 400 = {\color{blue}\boxed{400}}
    \end{align*}

  \end{enumerate}

  \linejump
  \item La gerencia de un banco debe decidir si instalará o no un sistema de apoyo a decisiones para prestamos comerciales (un sistema de información gerencial en línea) que ayude a sus analistas a tomar este tipo de decisiones, La experiencia anterior indica que $X$, el número adicional (por año) de decisiones de préstamos correctas — aceptar buenas solicitudes de préstamos y rechazar las que tarde o temprano no podrán pagarse — atribuible al sistema de apoyo de decisiones, y $Y$ la duración (en años) de este sistema, tienen la distribución de probabilidad conjunta que se muestra en la tabla.
  \end{enumerate}

  \newpage
  \begin{table}[h!]
    \centering
    \begin{tabular}{|cccccccccccc|}
    \hline
    \multicolumn{12}{|c|}{x}                                                                                                                                                                                                                                                                                                                       \\ \hline
    \multicolumn{2}{|c|}{}                                            & \multicolumn{1}{c|}{0}     & \multicolumn{1}{c|}{10}    & \multicolumn{1}{c|}{20}    & \multicolumn{1}{c|}{30}    & \multicolumn{1}{c|}{40}    & \multicolumn{1}{c|}{50}    & \multicolumn{1}{c|}{60}    & \multicolumn{1}{c|}{70}    & \multicolumn{1}{c|}{80}    & 90    \\ \hline
    \multicolumn{1}{|c|}{\multirow{5}{*}{y}} & \multicolumn{1}{c|}{1} & \multicolumn{1}{c|}{0.001} & \multicolumn{1}{c|}{0.002} & \multicolumn{1}{c|}{0.002} & \multicolumn{1}{c|}{0.025} & \multicolumn{1}{c|}{0.040} & \multicolumn{1}{c|}{0.025} & \multicolumn{1}{c|}{0.005} & \multicolumn{1}{c|}{0.005} & \multicolumn{1}{c|}{0}     & 0     \\ \cline{2-12} 
    \multicolumn{1}{|c|}{}                   & \multicolumn{1}{c|}{2} & \multicolumn{1}{c|}{0.005} & \multicolumn{1}{c|}{0.005} & \multicolumn{1}{c|}{0.010} & \multicolumn{1}{c|}{0.075} & \multicolumn{1}{c|}{0.100} & \multicolumn{1}{c|}{0.075} & \multicolumn{1}{c|}{0.050} & \multicolumn{1}{c|}{0.030} & \multicolumn{1}{c|}{0.030} & 0.025 \\ \cline{2-12} 
    \multicolumn{1}{|c|}{}                   & \multicolumn{1}{c|}{3} & \multicolumn{1}{c|}{0}     & \multicolumn{1}{c|}{0}     & \multicolumn{1}{c|}{0}     & \multicolumn{1}{c|}{0.025} & \multicolumn{1}{c|}{0.050} & \multicolumn{1}{c|}{0.080} & \multicolumn{1}{c|}{0.050} & \multicolumn{1}{c|}{0.080} & \multicolumn{1}{c|}{0.040} & 0.030 \\ \cline{2-12} 
    \multicolumn{1}{|c|}{}                   & \multicolumn{1}{c|}{4} & \multicolumn{1}{c|}{0}     & \multicolumn{1}{c|}{0.001} & \multicolumn{1}{c|}{0.002} & \multicolumn{1}{c|}{0.005} & \multicolumn{1}{c|}{0.010} & \multicolumn{1}{c|}{0.025} & \multicolumn{1}{c|}{0.010} & \multicolumn{1}{c|}{0.003} & \multicolumn{1}{c|}{0.001} & 0.001 \\ \cline{2-12} 
    \multicolumn{1}{|c|}{}                   & \multicolumn{1}{c|}{5} & \multicolumn{1}{c|}{0}     & \multicolumn{1}{c|}{0.002} & \multicolumn{1}{c|}{0.005} & \multicolumn{1}{c|}{0.005} & \multicolumn{1}{c|}{0.020} & \multicolumn{1}{c|}{0.030} & \multicolumn{1}{c|}{0.015} & \multicolumn{1}{c|}{0}     & \multicolumn{1}{c|}{0}     & 0     \\ \hline
    \end{tabular}
  \end{table}

  \linejump
  \begin{enumerate}
    \item Calcule las distribuciones de probabilidad marginal $p_X(x)$ y $p_Y(y)$
    
    \begin{align*}
      &p_X(0) = \sum_{y=1}^5 p_{XY}(0, y) = 0.001 + 0.005 + 0 + 0 + 0 = 0.006 \\
      &p_X(10) = \sum_{y=1}^5 p_{XY}(10, y) = 0.002 + 0.005 + 0 + 0.001 + 0.002 = 0.010 \\
      &p_X(20) = \sum_{y=1}^5 p_{XY}(20, y) = 0.002 + 0.010 + 0 + 0.002 + 0.005 = 0.019 \\
      &p_X(30) = \sum_{y=1}^5 p_{XY}(30, y) = 0.025 + 0.075 + 0.025 + 0.005 + 0.005 = 0.135 \\
      &p_X(40) = \sum_{y=1}^5 p_{XY}(40, y) = 0.040 + 0.100 + 0.050 + 0.010 + 0.020 = 0.220 \\
      &p_X(50) = \sum_{y=1}^5 p_{XY}(50, y) = 0.025 + 0.075 + 0.080 + 0.025 + 0.030 = 0.235 \\
      &p_X(60) = \sum_{y=1}^5 p_{XY}(60, y) = 0.005 + 0.050 + 0.050 + 0.010 + 0.015 = 0.130 \\
      &p_X(70) = \sum_{y=1}^5 p_{XY}(70, y) = 0.005 + 0.030 + 0.080 + 0.003 + 0 = 0.118 \\
      &p_X(80) = \sum_{y=1}^5 p_{XY}(80, y) = 0 + 0.030 + 0.040 + 0.001 + 0 = 0.071 \\
      &p_X(90) = \sum_{y=1}^5 p_{XY}(90, y) = 0 + 0.025 + 0.030 + 0.001 + 0 = 0.056 \\
      &p_X(x) = \sum_{y=1}^5 p_{XY}(x, y) = 0.006 + 0.010 + 0.019 + 0.135 + 0.220 + 0.235 + 0.130 + 0.118 \\
      &\hspace*{1.5cm} + 0.071 + 0.056 = 1
    \end{align*}

    \begin{align*}
      &p_Y(1) = \sum_{x=0}^{90} p_{XY}(x, 1) = 0.001 + 0.002 + 0.002 + 0.025 + 0.040 + 0.025 + 0.005 \\
      &\hspace*{1.5cm} + 0.005 + 0 + 0 = 0.105 \\
      &p_Y(2) = \sum_{x=0}^{90} p_{XY}(x, 2) = 0.005 + 0.005 + 0.010 + 0.075 + 0.100 + 0.075 + 0.050 \\
      &\hspace*{1.5cm} + 0.030 + 0.030 + 0.025 = 0.405 \\
      &p_Y(3) = \sum_{x=0}^{90} p_{XY}(x, 3) = 0 + 0 + 0 + 0.025 + 0.050 + 0.080 + 0.050 + 0.080 + 0.040 \\
      &\hspace*{1.5cm} + 0.030 = 0.355 \\
      &p_Y(4) = \sum_{x=0}^{90} p_{XY}(x, 4) = 0 + 0.001 + 0.002 + 0.005 + 0.010 + 0.025 + 0.010 + 0.003 \\
      &\hspace*{1.5cm} + 0.001 + 0.001 = 0.058 \\
      &p_Y(5) = \sum_{x=0}^{90} p_{XY}(x, 5) = 0 + 0.002 + 0.005 + 0.005 + 0.020 + 0.030 + 0.015 + 0 + 0 \\
      &\hspace*{1.5cm} + 0 = 0.077 \\
      &p_Y(y) = \sum_{x=0}^{90} p_{XY}(x, y) = 0.105 + 0.405 + 0.355 + 0.058 + 0.077 = 1
    \end{align*}

    \item Obtenga la distribución de probabilidad condicional $p_X(x \mid y)$
    
    \item Dado que el sistema de apoyo a decisiones está en su tercer año de operación, calcule la probabilidad de que se tomarán por lo menos 40 decisiones de préstamos correctas adicionales.
    
    \item Calcule la duración esperada del sistema de apoyo a decisiones, es decir, calcule $E[Y]$
    
    \item ¿Hay correlación entre $X$ y $Y$? ¿Son $X$ y $Y$. independientes?
    
    \item Cada decisión de préstamo correcta aporta aproximadamente 25,000 dólares a las utilidades del banco: calcule la media y la desviación estándar de las utilidades adicionales atribuibles al sistema de apoyo a decisiones. (Sugerencia: utilice la distribución marginal $p_X(x)$)
    
  \end{enumerate}
  
\end{document}