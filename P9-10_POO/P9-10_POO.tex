\documentclass[11pt, twocolumn]{article}

\usepackage[spanish]{babel}
\usepackage[none]{hyphenat}
\usepackage[left=1.2cm, right=1.2cm, top = 2cm, bottom=2.5cm]{geometry}
% \usepackage{setspace}
\usepackage{parskip}
\usepackage[export]{adjustbox}
\usepackage{enumitem}
\usepackage{listingsutf8}
\usepackage[dvipsnames]{xcolor}
\usepackage{fancyhdr}
\usepackage{graphicx}
\usepackage{caption}
% \usepackage{subcaption}
% \usepackage{wrapfig}
% \usepackage{multirow, makecell}
% \usepackage{float}
% \usepackage{amsmath} 
% \usepackage{amsfonts}
\usepackage[hidelinks]{hyperref}
\usepackage{csquotes}

\newcommand{\linejump}{\hfill \break}
\renewcommand{\thefootnote}{\fnsymbol{footnote}}
% \newcommand{\unit}[1]{\ensuremath{\, \mathrm{#1}}}

\definecolor{dkgreen}{rgb}{0,0.6,0}
\definecolor{gray}{rgb}{0.5,0.5,0.5}
\definecolor{mauve}{rgb}{0.58,0,0.82}
\lstset{
  language=Java,
  aboveskip=3mm,
  belowskip=3mm,
  showstringspaces=false,
  columns=flexible,
  basicstyle={\tiny\ttfamily},
  numbers=none,
  numberstyle=\tiny\color{gray},
  keywordstyle=\color{blue},
  commentstyle=\color{dkgreen},
  stringstyle=\color{mauve},
  breaklines=true,
  breakatwhitespace=true,
  tabsize=2
}

\sloppy
\setlength{\parindent}{0cm}
\setlength{\columnsep}{0.5cm}
\decimalpoint
\graphicspath{{img/}}

\hypersetup{colorlinks=true, urlcolor=blue, citecolor=blue}
\urlstyle{same}

\pagestyle{fancyplain}
\fancyhf{}
\fancyhead[L]{\scriptsize 
  Universidad Nacional Autónoma de México \\
  Laboratorio de Programación Orientada a Objetos \\
  M.C. Leonardo Ledesma Dominguez
}
\fancyhead[R]{\thepage}

\begin{document}
  \twocolumn[
    \centering
    Acosta Porcayo Alan Omar, Gutiérrez Grimaldo Alejandro, Medina Villa Samuel

    \linejump

    \textbf{\LARGE{Práctica 9. UML}} \\
    
    \linejump
  ]
      
  \footnotetext{
    \scriptsize 
    Acosta Porcayo Alan Omar Ing. en Computación 320206102 \\
    Gutiérrez Grimaldo Alejandro Ing. en Computación 320282098 \\
    Medina Villa Samuel Ing. en Computación 320249538
  }
        
  \fancyfoot{}

  \section*{Resumen}
  

  \section*{Introducción}


  \section*{Objetivos}
  \begin{itemize}
    \item 
  \end{itemize}

  \section*{Resultados}
  \subsection*{Problema 1}
  Agregue el esquema UML que le tocó realizar en la práctica 9 de laboratorio y de una breve explicación.


  \subsection*{Problema 2}
  Seleccione un diagrama UML de otro equipo de laboratorio, anéxelo a su reporte y de una breve explicación.


  \section*{Conclusiones}


  \twocolumn[
    \centering
    Acosta Porcayo Alan Omar, Gutiérrez Grimaldo Alejandro, Medina Villa Samuel

    \linejump
    
    \textbf{\LARGE{Práctica 10. Excepciones y errores}} \\

    \linejump
  ]

  \section*{Resumen}
  

  \section*{Introducción}


  \section*{Objetivos}
  \begin{itemize}
    \item 
  \end{itemize}

  \section*{Metodología}


  \section*{Resultados}
  \subsection*{Problema 1}
  Modifique el programa visto durante la práctica para construir objetos ``Persona'' desde consola solicitando nombre, edad, nacionalidad y sexo, la llave CURP.


  \subsection*{Problema 2}
  Revise el siguiente código ubicado en: \url{https://howtodoinjava.com/java/exception-handling/best-practices-for-for-exception-handling/}

  Implemente el punto 2 del blog utilizando la misma lógica para controlar excepciones:

  \begin{enumerate}[label=\alph*)]
    \item Un extranjero no puede ser mexicano, al momento de ser creado.
    \item Construya una excepción que no permita crear Personas con edad $<=$ 0 años.
    \item Construya una excepción que permita jubilar a una Persona considere la edad de jubilación de $>=$ 64 años.
  \end{enumerate}
  

  \subsection*{Problema 3}
  Usando instrucciones \textit{try-catch} aplique la jerarquía de excepciones de la siguiente manera:

  \begin{enumerate}
    \item Es mexicano
    \item Puede Votar
    \item Puede ser presidente
  \end{enumerate}

  Considerando que la 1 es la de mas alta prioridad y podrá acceder al nivel 2 y así sucesivamente.


  \section*{Conclusiones}


  \section*{Referencias}
  \begin{small}
    Gupta, L., $\&$ Gupta, L. (2023, April 7). \textit{Effective approach for creating custom exceptions in Java}. HowToDoInJava. \url{https://howtodoinjava.com/java/exception-handling/best-practices-for-for-exception-handling/} \\

    Solano, J. (2017, 20 enero). \textit{Manual de prácticas de Programación Orientada a Objetos}. Laboratorio de Computación Salas A y B. \url{http://lcp02.fi-b.unam.mx/} \\
  \end{small}
\end{document}