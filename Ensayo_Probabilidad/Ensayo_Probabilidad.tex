\documentclass{article}[12pt]

\usepackage[spanish]{babel}
\usepackage[none]{hyphenat}
\usepackage[margin=3cm]{geometry}
\usepackage{enumitem} 
\usepackage{fancyhdr}
\usepackage{setspace}
\usepackage[hidelinks]{hyperref}
\usepackage{parskip}

\sloppy
\onehalfspacing
\pagestyle{fancy}
\setlength{\parindent}{0cm}
\decimalpoint
\fancyhead[R]{Acosta Porcayo Alan Omar}
\hypersetup{colorlinks=true, urlcolor=blue, citecolor=blue}
\urlstyle{same}

\begin{document}  
    \begin{center}
        \large \textbf{¿Determinismo, azar o probabilidad?}
    \end{center}
    
    A lo largo de la historia la humanidad se ha hecho la pregunta de si todas las acciones están determinadas o hay un margen para el azar y el libre albedrio. Desde entonces, el hombre científico inició la especulación filosófica para conocer los orígenes y elementos constitutivos de la naturaleza, de tal manera que se encontraran los principios universales.

    La naturaleza estaba dotada de una armonía interna y, lo que es más, que era una estructura organizada. Los ejemplos astronómicos y geométricos demostraron que era posible llegar al descubrimiento de las verdades ocultas. Gracias a científicos como Bacon, Leibniz, Newton, Copérnico, Galileo o Kepler; Las leyes de la naturaleza adquirieron una forma matemática explícita, y, así, la relación entre los conceptos de la teoría y los elementos de la realidad física permitían avanzar en el conocimiento del sistema al que pertenecían.

    Una pregunta apareció, ¿el hombre formaba parte de las leyes universales?.  Los nacimientos o defunciones mostraban una relación que no podía sino ser producto del mismo orden natural al que pertenecían los hechos físicos. Esto indicaba que existía una ley aplicable al hombre, por lo que los actos humanos estaban determinados. Las regularidades eran una mera expresión de las leyes de la naturaleza a las que estaban supeditadas. Sin embargo, el comportamiento de una sociedad es completamente diferente al de los átomos o moléculas del mundo físico. Las leyes de la naturaleza se refieren a sistemas físicos, no al comportamiento humano.

    La mecánica clásica, se había establecido firmemente en el siglo XVII. El científico clásico encontraba, medía y consideraba al mundo tal como era. Pero Si en toda medición existía un margen de error, entonces había fenómenos que escapaban a su precisión. 

    Si en un pasado el azar fue mirado con desprecio, a partir de los inicios del siglo XX fue visto de manera diferente.  Fue entonces que la mecánica cuántica había surgido para describir el mundo microscópico. Según los principios de la mecánica cuántica, toda descripción del mundo no es objetiva ni determinista.  La totalidad de los eventos está determinada por las leyes de la probabilidad, de las que para un estado en el espacio corresponde una de estas probabilidad.

    Son leyes probabilísticas las que determinan nuestro conocimiento de los elementos constitutivos del universo. Ahora el azar ha sido asociado a los fenómenos que no tienen ley, fenómenos cuyas causas son en extremo complejas para nuestro entendimiento. La *La Ley de los grandes números* de Poisson se convirtió así en un legado que la mecánica cuántica adoptó para hacer comprensibles las "leyes naturales".

    Nuestra vida depende de las probabilidades. La mayor parte de los fenómenos que nos rodean forman parte de ese mundo del azar, y que la probabilidad es tan sólo la expresión de ese universo.
    
    \newpage
    \begin{center}
        \large \textbf{Como la estadística engaña a los jurados}
    \end{center}

    El video trata el tema de como los problemas que tienen que ver con la estadística son poco intuitivos para el publico general y en ocasiones pueden conllevar a malas interpretaciones. 

    El primer ejemplo que pone el conferencista es el promedio de veces hasta que una moneda necesita ser lanzada hasta que caiga en un patrón definido. El primero es cara-cruz-cruz y el segundo es cara-cruz-cara, entonces presenta 3 opciones: el primer patrón es mas probable, los dos son igual de probables o el segundo es más probable. La mayoría del público prefiere la segunda opción, sin embargo la correcta es la primera, pues en promedio el patrón cara-cruz-cruz tarda 8 lanzamientos y el cara-cruz-cara tarda 10 lanzamientos. 

    Por otro lado, el conferencista habla de como la genética y la probabilidad están relacionadas y menciona el proyecto del Genoma Humano y  el proyecto Internacional HapMap en los que participó.  El primero buscaba analizar los elementos que las personas tienen en común, y el segundo las que son diferentes. Estas investigaciones permitirían saber las diferencias que hacen a las personas ser propensas a enfermedades.

    Otro ejemplo que pone el matemático es de un test con 99\% de acierto y se le aplica a una persona, si el test marca positivo ¿cuál es la posibilidad de que la persona realmente tenga la enfermedad?. La respuesta natural seria 99\% pero realmente depende de que tan común sea la enfermedad. 

    Un ejemplo más es como en Inglaterra sucedió un juicio contra una mujer que perdió dos hijos por "muerte de cuna" y fue considerada culpable tomando como evidencia la opinión de un pediatra sobre la probabilidad de dos muertes de cuna en una familia como la suya, la cual fue de 73 millones. 

    El error que tuvieron estos cálculos fue tomar a los eventos como independientes que no se conocen, a esto se le conoce en la estadística como una suposición de independencia. Después del juicio un periodista escribió que "la probabilidad de que la acusada sea inocente es de una en 73 millones", lo que es una mala interpretación de la estadística. En situaciones como la anterior, los errores en la estadística tuvieron profundas consecuencias.

    El azar, la probabilidad y la incertidumbre forman parte de la vida diaria, sin embargo, es común cometer errores de lógica al razonar con la incertidumbre, y al menos deberíamos ser conscientes de ello.
    
    \hfil \break
    \begin{flushleft}
        \large\textbf{Bibliografía}
    \end{flushleft}

    Flores de la Cruz, M. E. (s. f.). ¿Determinismo, azar o probabilidad? Universidad Veracruzana. Repositorio Institucional. \url{https://cdigital.uv.mx/bitstream/handle/123456789/5514/20031P17.pdf?sequence=2&isAllowed=y}

    \hfil \break
    TED. (2007, January 12). Peter Donnelly: How stats fool juries [Video]. YouTube. \url{https://www.youtube.com/watch?v=kLmzxmRcUTo}
\end{document}