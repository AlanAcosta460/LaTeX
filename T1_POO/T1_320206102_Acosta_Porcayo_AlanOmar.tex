\documentclass[12pt]{article}

\usepackage[spanish]{babel}
\usepackage[none]{hyphenat}
\usepackage[margin=3cm]{geometry}
\usepackage{enumitem}
\usepackage{longtable}
\usepackage{multirow, makecell}
\usepackage{listings}
\usepackage{color}

\definecolor{dkgreen}{rgb}{0,0.6,0}
\definecolor{gray}{rgb}{0.5,0.5,0.5}
\definecolor{mauve}{rgb}{0.58,0,0.82}

\lstset{
    language=Java,
    aboveskip=3mm,
    belowskip=3mm,
    showstringspaces=false,
    columns=flexible,
    basicstyle={\scriptsize\ttfamily},
    numbers=none,
    numberstyle=\tiny\color{gray},
    keywordstyle=\color{blue},
    commentstyle=\color{dkgreen},
    stringstyle=\color{mauve},
    breaklines=true,
    breakatwhitespace=true,
    tabsize=3
}

\sloppy
\setlength{\parindent}{0cm}
\decimalpoint
\setenumerate[2]{label=\alph*)}

\begin{document}
    \begin{center}
        \large \textbf{Elementos y propiedades básicos de POO}
    \end{center}
    
    \begin{flushright}
        32020610-2. \\
        Acosta Porcayo Alan Omar.
    \end{flushright}

    \begin{enumerate}
        \item Investigue los siguientes conceptos relacionados a POO y describa con sus propias palabras la definición de cada uno de ellos:
        \begin{enumerate}
            \item Abstracción:
            \item Polimorfismo:
            \item Herencia:
            \item Encapsulamiento:
        \end{enumerate}
        Dé ejemplos triviales de la aplicación de estos cuatro conceptos.

        \item Investigue que otro tipo de paradigmas de programación existen y haga una tabla de ventajas y desventajas con respecto a POO.
        % \begin{longtable}
        % \end{longtable}

        \item Realice un programa en JAVA que calcule la suma de los primeros n números naturales, donde n sea un parámetro que se recibe del teclado. 
        \begin{lstlisting}
public class suma_n_naturales {
    public static void main(String[] args) {
        Scanner sc = new Scanner(System.in);

        System.out.print("Ingrese el valor de n: ");
        int n = sc.nextInt(), suma = 0;
        sc.close();

        for(int i = 1; i <= n; i ++) {
            suma += i;
        }

        System.out.println("El valor de la suma de los primeros " + n + " naturales es: " + suma);

    }
}
        \end{lstlisting}

        \item Realice un programa en JAVA que calcule la suma de los primeros n números naturales, donde n sea un parámetro que se recibe del teclado. 
        \begin{lstlisting}
import java.util.Scanner;

public class factorial {
    public static void main(String[] args) {
        Scanner sc = new Scanner(System.in);
    
        System.out.print("Ingrese el valor de n para calcular su factorial: ");
        int n = sc.nextInt(), factorial = 1;
        sc.close();

        for(int aux = n; aux > 1; aux --) {
            factorial *= aux;
        }

        System.out.println("El factorial de " + n + " es: " + factorial);
    }
}
        \end{lstlisting}

        \item Expresa tu comentario del video sobre “El fin de los programadores”
    \end{enumerate}
\end{document}
